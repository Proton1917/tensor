\documentclass{article}
\usepackage{mathrsfs}
\usepackage{amsfonts}
\usepackage{amsmath}
\title{Representation of Vectors and Tensors}
\author{Neutron}
\date{\today}

\begin{document}

\maketitle
\section*{Reading Instructions for This Document}

This document assumes that the reader has prior knowledge of calculus and linear algebra. Additionally, the reader should be familiar with the Einstein summation convention, which is used extensively throughout the text.
The Einstein summation convention is a notational shorthand where repeated indices in a mathematical expression imply summation over those indices. For example, the expression

\[
a_\mu b^\mu
\]

implies the sum

\[
\sum_{\mu=1}^n a_\mu b^\mu
\]

where \( \mu \) runs over all possible values. This convention is widely used in tensor calculus to simplify the notation of vector and tensor operations.

\section{Tangent Vectors and Vector Space}
\subsection{Definition}

A mapping \( v: \mathscr{F}_M \rightarrow \mathbb{R}\) is called a vector at a point \( p \in M \),if  \(\forall f,g \in \mathscr{F}_M , \alpha,\beta \in \mathbb{R}\)\\
(a)(linearity)  \(v(\alpha f+\beta g)=\alpha v(f)+\beta v(g)\);\\
(b)(Leibniz’s Rule)  \(v(fg)=f|_p \,v(g)+g|_p \,v(f)\)
\subsection{Coordinate Basis and Coordinate Components}

A coordinate basis \(\{e_1, \dots, e_n\}\) of a vector space can provide a way to express vectors. Each \(e_\mu\) is called a coordinate basis vector, and we generally define:\\
\(e_\mu = \dfrac{\partial}{\partial x^\mu}\),Let \(\{ x^\mu\} \) and \( \{x'^\nu\} \) be two coordinate systems, and \( p \) be a point in these systems. The components \( v^\nu \) and \( v'^\mu \) are the coordinates of the vector \( v \) in these two systems, respectively. The relationship between these components is given by:
\[
v'^\mu = \dfrac{\partial x’^\mu}{\partial x^\nu}\bigg|_p v^\nu 
\]
\section{Dual Vectors and Dual Space}
\subsection{Definition and Property}

A dual vector (or covector) at a point \( p \in M \) is defined as a linear map \( \omega: V \rightarrow \mathbb{R} \), where \( V \) is the vector space associated with the point \( p \). This map assigns to each vector \( v \in V \) a real number \( \omega(v) \), satisfying the linearity property:

\begin{itemize}
    \item (Linearity) \( \omega(\alpha v + \beta w) = \alpha \omega(v) + \beta \omega(w) \) for all vectors \( v, w \in V \) and scalars \( \alpha, \beta \in \mathbb{R} \).
\end{itemize}

The dual vector \( \omega \) can be expressed in terms of a coordinate basis \( \{e_1, \dots, e_n\} \) for the vector space \( V \). The corresponding dual basis \(\{e^{\mu*}\}\) is defined such that \[ e^{\mu*}(e_\nu) = \delta^\mu_{\phantom{\mu}\nu} ,\qquad \mu,\nu =1,\ldots,n\]where \( \delta^\mu_{\phantom{\mu}\nu} \)is the Kronecker delta. The dual vector \( \omega \) can then be written as:

\[
\omega = \omega_\mu \, e^{\mu*}
\]
We define \( e^{\mu*} = dx^\mu \) as the dual basis corresponding to the coordinate basis \( \{e_\mu\} \). This dual basis satisfies the condition:

\[
\mathrm{d}x^\mu(\dfrac{\partial}{\partial x^\nu}) = \dfrac{\partial}{\partial x^\nu}(\mathrm{d}x^\mu)=\delta^\mu_{\phantom{\mu} \nu}
\]
where \( \omega_\mu \) are the components of the dual vector in the chosen coordinate system.

The relationship between the components of a dual vector \( \omega \) in two different coordinate systems \( \{ x^\mu\} \) and \( \{ x'^\nu\} \) is given by:

\[
\omega'_\mu = \frac{\partial x^\nu}{\partial x'^\mu} \bigg|_p\omega_\nu
\]
Here, \( \omega_\nu \) and \( \omega'_\mu \) are the components of the dual vector \( \omega \) in the original and transformed coordinate systems, respectively. This transformation rule ensures that the dual vector behaves covariantly under a change of coordinates.
\subsection{Contravariant and Covariant Vectors}

A contravariant vector \( v=v^\mu e_\mu \) is simply a vector in the vector space \( V \), while a covariant vector \( \omega=\omega_\nu e^{\nu*}\) is a dual vector in the dual space \( V^* \).\\

The components of a vector(contravariant vector) are denoted with upper indices (e.g., \(v^\mu\)), and the basis vectors are denoted with lower indices (e.g., \(e_\mu\)). Conversely, for dual vectors(covariant vector), the components are denoted with lower indices (e.g., \(\omega_\nu\)), and the dual basis vectors are denoted with upper indices (e.g., \(e^{\nu*}\)).
\section{Tensors}
\subsection{Definition}
A tensor of type \((k, l)\) on a vector space \( V \) is defined as a multilinear map:

\[
T: \underbrace{V^* \times \dots \times V^*}_{k \text{ copies}} \times \underbrace{V \times \dots \times V}_{l \text{ copies}} \rightarrow \mathbb{R}
\]

This means that \( T \) takes \( k \) covariant vectors (or dual vectors) and \( l \) contravariant vectors (or vectors) as arguments and returns a real number, satisfying the property of multilinearity. That is, \( T \) is linear in each of its arguments.
\subsection{Tensor Product on \( V \)}

The tensor product of two tensors \( T_1 \) and \( T_2 \) on a vector space \( V \) is a new tensor, denoted \( T_1 \otimes T_2 \). If \( T_1 \) is a tensor of type \((k_1, l_1)\) and \( T_2 \) is a tensor of type \((k_2, l_2)\), then the tensor product \( T_1 \otimes T_2 \) is a tensor of type \((k_1 + k_2, l_1 + l_2)\). The tensor product is defined by the following bilinear map:
\[
(T_1 \otimes T_2)(\omega^1, \dots, \omega^{k_1+k_2}; v_1, \dots, v_{l_1+l_2})\]
\[: = T_1(\omega^1, \dots, \omega^{k_1}; v_1, \dots, v_{l_1}) \cdot T_2(\omega^{k_1+1}, \dots, \omega^{k_1+k_2}; v_{l_1+1}, \dots, v_{l_1+l_2})
\]

Here, \( \omega^i \) are covectors (elements of \( V^* \)), and \( v_i \) are vectors (elements of \( V \)). The operation \( \otimes \) creates a new tensor that acts on the combined set of covectors and vectors from the original tensors.

In a given basis, the components of a \( (2,1) \) tensor \( T \) can be expressed as \( T^{\mu\nu}_{\phantom{\mu\nu}\lambda} \). The tensor \( T \) in this basis is represented as:

\[
T = T^{\mu\nu}_{\phantom{\mu\nu}\lambda} \, e_\mu \otimes e_\nu \otimes e^\lambda
\]

\noindent where \( e_\mu \) and \( e_\nu \) are the basis vectors, and \( e^\lambda \) is the dual basis vector. The component \( T^{\mu\nu}_{\phantom{\mu\nu}\lambda} \) is defined as \( T^{\mu\nu}_{\phantom{\mu\nu}\lambda} = T(e^{\mu*} , e^{\nu*} ; e_\sigma) \). 
\subsection{Metric Tensor}

The metric tensor \( g \) is a fundamental object in differential geometry and general relativity. It is a symmetric, bilinear form that defines the distance between two points in a given space. The metric tensor allows for the calculation of the length of vectors, the angle between vectors, and the inner product of vectors.Here is the fundamental express of metric tensor:
\[g(v,u)=g(u,v)
\]

Mathematically, the components of metric tensor \( g \) is defined as:

\[
g_{\mu\nu} = g\left(\frac{\partial}{\partial x^\mu}, \frac{\partial}{\partial x^\nu}\right)
\]

where \( \dfrac{\partial}{\partial x^\mu} \) and \( \dfrac{\partial}{\partial x^\nu} \) are basis vectors in the tangent space at a point. The components \( g_{\mu\nu} \) are the entries of the metric tensor in a given coordinate system, and they determine the geometry of the space.

The metric tensor also allows for the raising and lowering of indices on tensors, thereby converting between contravariant and covariant components:

\[
v_\mu = g_{\mu\nu} v^\nu
\quad \text{and} \quad
v^\mu = g^{\mu\nu} v_\nu
\]

In these expressions, \( g^{\mu\nu} \) represents the inverse of the metric tensor \( g_{\mu\nu} \), which satisfies \( g_{\mu\alpha} g^{\alpha\nu} = \delta^\nu_\mu \), where \( \delta^\nu_\mu \) is the Kronecker delta.
\section{Abstract Index Notation}

Abstract index notation is a formalism in tensor calculus and differential geometry that allows for the manipulation of tensors in a way that is independent of any specific coordinate system. This notation emphasizes the structural and algebraic properties of tensors rather than their specific numerical components.

\subsection{Abstract Indices: Usage, Motivation, and Examples}

Abstract indices are symbolic placeholders used to denote the type and rank of tensors without assigning them specific numerical values. These indices are typically represented by lowercase Latin letters, starting from \( a \) and continuing up to the letter before \( i \) (e.g., \( a, b, c, \dots, h \)). These letters are used purely to indicate the tensor’s structure, such as whether it is contravariant or covariant, without reference to a particular coordinate system.

For example, a tensor \( T^{ab}_{\phantom{ab}c} \) is a type \((2,1)\) tensor with two contravariant indices \( a \) and \( b \), and one covariant index \( c \). The abstract index notation allows us to express the properties and operations of this tensor, such as tensor products and contractions, without specifying the components.

The motivation behind the invention of abstract index notation is to facilitate a coordinate-free approach to tensor analysis, which is essential in various areas of mathematics and physics. By using abstract indices, one can manipulate tensor equations in a general way, applying them to any coordinate system.

Contraction, an operation that reduces the rank of a tensor, is expressed using abstract indices by summing over an upper and a lower index. For instance, contracting the tensor \( T^{ab}_{\phantom{ab}b} \) results in:

\[
T^a = T^{ab}_{\phantom{ab}b}
\]

Here, the index \( b \) appears both as an upper and a lower index, indicating the contraction.

\subsection{Concrete Indices: Tensor Components with Numerical Values}

Concrete indices, unlike abstract indices, refer to the actual numerical components of a tensor in a given coordinate system. These indices are typically represented by lowercase Greek letters such as \( \mu, \nu, \lambda \), and they correspond to specific values depending on the dimensions of the space.

For example, in a 4-dimensional spacetime, a vector \( v^a=v^\mu(e_\mu)^a \) might have components \( v^0, v^1, v^2, v^3 \).
When transitioning from abstract indices to concrete indices, a tensor \( T^{ab}_{\phantom{ab}c} \) can be expanded in terms of its components as:
\[T^{ab}_{\phantom{ab}c} = T^{\mu\nu}_{\phantom{\mu\nu}\lambda} (e_\mu)^a (e_\nu)^b (e^\lambda)_c
\]
Here, the abstract indices \( a, b, c \) have been replaced by the specific concrete indices \( \mu, \nu, \lambda \), where each \( \mu, \nu, \lambda \) corresponds to a specific coordinate in the chosen basis. The components \( T^{\mu\nu}_{\phantom{\mu\nu}\lambda} \) represent the actual numerical values of the tensor in this coordinate system.
Concrete indices are essential for performing explicit calculations in physics and geometry, as they provide the actual data needed to solve problems or make predictions.
At this point, we can change the basis to one consisting of tangent vectors and cotangent vectors. For example, instead of using the abstract basis \( (e_\mu)^a \), we can express the tensor in terms of the basis \( \left(\dfrac{\partial}{\partial x^\mu}\right)_a \) for  vectors and \( (\mathrm{d}x^\lambda)^c \) for dual vectors:

\[
T^{ab}_{\phantom{ab}c} = T^{\mu\nu}_{\phantom{\mu\nu}\lambda} \left(\frac{\partial}{\partial x^\mu}\right)^a \left(\frac{\partial}{\partial x^\nu}\right)^b (\mathrm{d}x^\lambda)_c
\]
\section{Conclusion}

In this document, we explored various fundamental concepts in tensor calculus and differential geometry, starting with the definition and properties of vectors and dual vectors. We introduced the concept of coordinate basis and demonstrated how vectors and dual vectors can be expressed in different coordinate systems. The notion of tensors was then defined, including how tensors of different types are formed, and how their components can be computed in specific bases.

We discussed the metric tensor, a crucial object in defining geometric properties such as distances and angles in a given space. The metric tensor allows for the calculation of the length of vectors, the angle between vectors, and the inner product of vectors. Additionally, the metric tensor enables the raising and lowering of indices, allowing conversion between contravariant and covariant components.

Finally, we examined the abstract index notation, a powerful tool for representing tensors without relying on any particular coordinate system. This notation allows for more general and elegant manipulation of tensors, particularly when dealing with contractions and other tensor operations. By distinguishing between abstract and concrete indices, we can maintain a clear separation between the structural properties of tensors and their specific numerical components.

Overall, this document provides a foundational understanding of tensors, their properties, and the notations used to describe them, paving the way for more advanced studies in differential geometry, general relativity, and related fields.


\end{document}